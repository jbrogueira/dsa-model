In existing Debt Sustainability Analysis (DSA) methodologies -- including the one used by the European Commission \cite{DG-EFA_2025} within the new EU fiscal surveillance framework --, sovereign debt is determined by the evolution of primary surpluses and deficits, the input variable in these methods, possibly corrected by a ‘demographic cost’ as part of the risk assessment of a country. Nevertheless, current primary surpluses and deficits are the residual of accounting revenues and expenditures in the government budget and, looking forward, to the assets and liabilities that the government has. In fact, the currently non-fulfilled non-debt liabilities can become debt liabilities in the future.

Hence, looking forward, a broader perspective is needed, not only to account for long-term trends (e.g. population ageing, climate change), but also to account for the non-debt liabilities of the Public Sector Balance Sheet (PSBS) of the welfare state: public pensions, health and education, prevention of natural disasters. This also requires to account the investment in the public capital needed to support these services. To this long list of liabilities, an increase of up to 5\% of GDP is now expected for the EU (NATO) countries. However, the problem is not just to solve a difficult accounting arithmetic, but solve without jeopardizing the welfare of future generations. For instance, austerity in a crisis may prevent it from becoming a sovereign debt crisis, but possibly at the cost of having a less resilient economy (e.g. a depreciated public health system) to confront a future crisis, as ‘stressed countries’ during the euro crisis have later experienced with the COVID-19 crisis. That is, the promise that European citizens can have public health coverage is a public liability, that must be covered with tax revenues or debt, but its fulfillment also implies a more resilient economy; i.e. an economy where more risks are prevented or, if not, they are insured or less costly.

In sum, a ‘broader perspective’ is needed to properly assess the risk profile of a country. In fact, a problem of this century for many European economies (not only the ‘stressed countries’) has been their lack of proper counter-cyclical fiscal policy (e.g. not saving enough in good times). This is usually considered a time-inconsistency, in solving a consumption vs. saving problem. A ‘broader perspective’ brings new light into this problem. Not only because the social welfare liabilities are accounted for, but because the ‘broader problem’ becomes consumption and investment in public capital vs saving or borrowing. That is, with commitment one should invest enough in good times but not deplete public capital in bad times. Our analysis strengthens LSA with a proper stochastic dynamic model.
 
Surprisingly, while by now there is a very extensive applied and theoretical literature on sovereign debt crises (e.g. \cite{AguiarAmador2014}, \cite{Arellano_2008}, \cite{MendozaYue2012}, \cite{NiemPich_RED2020}, \cite{Park_JIE2017}), and the most advanced DSA methodologies build on it, PSBS data is scarce, and most theoretical models are silent regarding non-debt liabilities or public capital accumulation. In this project we plan to fill this gap, building on our previous work on fiscal policy, sovereign debt and social security reforms (with the ‘demographic ageing transition’) to study the future of fiscal policy and sovereign debt under this ‘broader perspective’, providing a framework to further explore Pareto improving policies which are robust to alternative model specifications. We develop a model where the government intertemporal choice is the ‘broader problem’. This provides, on the one hand, a better framework than current DSAs, to assess country risks and, on the other hand, a modelling toolkit to quantify macroeconomic policies and counterfactuals by calibrating a model to European economies, balancing detailed description of government expenditures, liabilities and revenues, with manageable aggregation of public capital and its risk-reduction effects. Next section describes it.

Our model builds on three strands of models: i) \cite{DiazSaavedraMarimonBrogueiradeSousa2023}, for social security liabilities and reforms with demographics; ii) \cite{AguiarAmador2014} et al. cited above, for debt default as particular case of liability default and to distinguish public capital from private capital, and iii) \cite{ACLM_2025}, \cite{LiuMarimonWichtJIE}, \cite{HolmMilg_JLEO91} and \cite{MarWicZav_2025}, for the design of policies, contracts and institutions in an economic union. 
