Sovereign debt is one of the liabilities of a government, but the list of non-debt liabilities is large, including the sustainability of the welfare state and the provision of public goods. Debt Sustainability Analyses (DSA) compares government debt projections to the evolution of budget surpluses but does not account for the intertemporal trade-offs of fiscal policy, when the objective is the long-run sustainability of government liabilities. Liability Sustainability Analysis (LSA) is needed, but to understand the trade-offs and design efficient sustainable and counter-cyclical fiscal policies a stochastic dynamic general equilibrium model is also needed. This analysis is particularly relevant now that European countries, already ageing and indebted, face an historical military buildup. There is a consensus that choices of policies and reforms will need to be made to keep all liabilities sustainable (e.g. \cite{IMF202511}), but not about which and how, and their welfare consequences. In this draft, we develop, theoretically and computationally, a dynamic overlapping generation model with heterogeneous cohorts that can encompass different forms of the European Welfare State, as well as of accumulation of private and public capital. Calibrated to different European countries, this model-framework will allow us: first, to have a better picture of the European \textquoteleft liabilities experiences\textquoteright\, through the XXI$^{\rm{st}}$ crises; second, address the question of whether their current liabilities are sustainable (and their welfare costs) in face of the ageaing transition and the foreseen military buildup, and third, study alternative policies and reforms that can make liabilities -- including debt -- sustainable, a guarantee to maintain, and possibly improve, European citizens welfare.  