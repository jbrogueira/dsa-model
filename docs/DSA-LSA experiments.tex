In this section we illustrate how the model can be used to study the aggregate and distributional effect of external changes. We analyze the economy's adjustment to a demographic change by simulating a deterministic transition over $T_{\text{trans}}$ = 50 periods. The economy starts in a time-invariant steady state and, at $t = 0$, begins transitioning to a new steady state. The two alternative transitions are driven by a decline in population growth from 1\% to 0\%, and by a decline in real rates from 4\% to 3\%.

Each experiment takes as inputs exogenous paths for prices and policy instruments (tax rates and pension replacement), solves the lifecycle problem for each birth cohort under perfect foresight about these paths, and aggregates across cohorts using demographic weights implied by a constant population growth rate (in the baseline scenario). The following plots report the model implied time paths of prices, main aggregates, and cohort-level lifecycle profiles (e.g., mean assets by age for cohorts born at different transition periods).

At this preliminary stage, we set model parameters ad hoc to conventional values used in the literature, and we do not yet discipline them using any formal estimation or calibration procedure.

% ------------------------------------------------------------------
% Subsection: Transition experiment — figures
% ------------------------------------------------------------------
\subsection{Transition experiment: figures}
\label{subsec:transition-figures}

This subsection reports the transition-path and cohort-level figures generated by the simulation. 
Figure~\ref{fig:transition_dynamics} plots aggregate transition dynamics, while 
Figure~\ref{fig:gov_budget} summarizes government budget components along the transition. 
Figure~\ref{fig:lifecycle_comparison} plot mean assets, consumption, incomes, and employment/unemployment rates by age for two cohorts (born at different transition periods) by education group.

\paragraph{Decline in population growth from 1\% to 0\%}
With the population growth rate declining exogenously from 1\% to 0\%, the transition exhibits slow-moving demographic reweighting that reduces effective labor supply over time and shifts the economy toward higher capital intensity: aggregate labor trends downward, output declines  alongside labor, and the capital–output ratio rises gradually as capital adjusts less than proportionally to labor. The lifecycle profiles are qualitatively unchanged across cohorts within each education group—assets accumulate during working ages and are run down after retirement, consumption is smoothed with a discrete adjustment at retirement, effective labor income drops to zero at retirement, and pension income becomes the primary source of resources thereafter—while cohort-to-cohort differences are small, reflecting largely stable individual decision rules under the demographic shock. On the fiscal side, expenditures drift upward (driven primarily by rising pension outlays as the age structure shifts), while revenues are broadly stable to slightly declining; as a result, the primary balance deteriorates gradually, and spending-to-GDP rises more than revenue-to-GDP over the transition, consistent with an aging-driven increase in dependency ratios.

\begin{figure}[H]
    \centering
    \includegraphics[width=\textwidth]{output/test/transition_20251220_220055_Ttr60_T40_nsim10000_highxlowxmedium.png}
    \caption{Transition dynamics: interest rate $r_t$, wage $w_t$, capital $K_t$, labor $L_t$, output $Y_t$, and $K_t/Y_t$.}
    \label{fig:transition_dynamics}
\end{figure}

\begin{figure}[H]
    \centering
    \includegraphics[width=\textwidth]{output/test/government_budget_20251220_220058_Ttr60_T40_nsim10000_highxlowxmedium.png}
    \caption{Government budget along the transition: revenues, expenditures (including pensions), and the primary deficit.}
    \label{fig:gov_budget}
\end{figure}

% \begin{figure}[H]
%     \centering
%     \includegraphics[width=\textwidth]{output/test/lifecycle_comparison_20251220_094545.png}
%     \caption{Mean assets by age for two cohorts (education: low). Lines are labeled by cohort birth period $t$ (the transition period in which the cohort is age 0).}
%     \label{fig:lifecycle_comparison_l}
% \end{figure}

\begin{figure}[H]
    \centering
    \includegraphics[width=\textwidth]{output/test/lifecycle_comparison_20251220_094546.png}
    \caption{Mean assets by age for two cohorts (education: medium). Lines are labeled by cohort birth period $t$ (the transition period in which the cohort is age 0).}
    \label{fig:lifecycle_comparison}
\end{figure}

% \begin{figure}[H]
%     \centering
%     \includegraphics[width=\textwidth]{output/test/lifecycle_comparison_20251220_094547.png}
%     \caption{Mean assets by age for two cohorts (education: high). Lines are labeled by cohort birth period $t$ (the transition period in which the cohort is age 0).}
%     \label{fig:lifecycle_comparison_h}
% \end{figure}

\paragraph{Decline in $r$ from 4\% to 3\%}

Over the transition, the exogenous decline in the interest rate from about 4\% to 3\% generates a smooth increase in wages and a gradual adjustment of aggregates: capital falls initially and then stabilizes, while aggregate labor remains essentially flat (with only small simulation noise), implying a mild decline and subsequent stabilization in output and in the capital–output ratio. The lifecycle comparisons across cohorts (born at ($t=0$) versus ($t=5$)) show very similar profiles within each education group: assets accumulate through working ages and are decumulated after retirement, consumption is relatively smooth with a discrete shift around retirement, effective labor income drops to zero at retirement, and pensions replace labor income thereafter; employment is near one during working life and zero after retirement, with low but nonzero UI recipiency concentrated in working ages. On the fiscal side, total revenues exceed total spending throughout the displayed periods, yielding a persistent primary surplus; labor and consumption taxes account for the bulk of revenues, while pensions are the largest spending component, and fiscal ratios (revenue/GDP and spending/GDP) are stable with only modest drift over time (with revenues tracking spending closely and the primary balance changing only slightly).
\begin{figure}[H]
    \centering
    \includegraphics[width=\textwidth]{output/test/transition_20251221_221053_Ttr60_T40_nsim10000_highxlowxmedium.png}
    \caption{Transition dynamics: interest rate $r_t$, wage $w_t$, capital $K_t$, labor $L_t$, output $Y_t$, and $K_t/Y_t$.}
    \label{fig:transition_dynamics}
\end{figure}

\begin{figure}[H]
    \centering
    \includegraphics[width=\textwidth]{output/test/government_budget_20251221_221056_Ttr60_T40_nsim10000_highxlowxmedium.png}
    \caption{Government budget along the transition: revenues, expenditures (including pensions), and the primary deficit.}
    \label{fig:gov_budget}
\end{figure}

% \begin{figure}[H]
%     \centering
%     \includegraphics[width=\textwidth]{output/test/lifecycle_comparison_20251220_094545.png}
%     \caption{Mean assets by age for two cohorts (education: low). Lines are labeled by cohort birth period $t$ (the transition period in which the cohort is age 0).}
%     \label{fig:lifecycle_comparison_l}
% \end{figure}

\begin{figure}[H]
    \centering
    \includegraphics[width=\textwidth]{output/test/lifecycle_comparison_20251221_221054_Ttr60_T40_nsim10000_highxlowxmedium.png}
    \caption{Mean assets by age for two cohorts (education: medium). Lines are labeled by cohort birth period $t$ (the transition period in which the cohort is age 0).}
    \label{fig:lifecycle_comparison_m}
\end{figure}

% \begin{figure}[H]
%     \centering
%     \includegraphics[width=\textwidth]{output/test/lifecycle_comparison_20251220_094547.png}
%     \caption{Mean assets by age for two cohorts (education: high). Lines are labeled by cohort birth period $t$ (the transition period in which the cohort is age 0).}
%     \label{fig:lifecycle_comparison_h}
% \end{figure}