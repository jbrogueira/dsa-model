\documentclass{article}
\usepackage{eurosym}
%\usepackage[pdftex]{graphics}
\usepackage{amsmath,amssymb,bbm,enumitem}
\usepackage{amsfonts,amsthm,mathrsfs,verbatim,bm}
\usepackage{import}
\DeclareMathOperator*{\argmax}{argmax}
\usepackage[authoryear,semicolon]{natbib}
\usepackage[toc,page]{appendix}
\usepackage{float}
\usepackage{graphicx}
\usepackage{colortbl}
\usepackage{xcolor}
\usepackage[caption = false]{subfig}
%\usepackage{caption,subcaption}
\usepackage{adjustbox}
\usepackage{threeparttable} 
\usepackage{booktabs,multirow}
\usepackage{color}
\usepackage{threeparttable}
\usepackage{booktabs}
\usepackage[bookmarks=true,            pdfstartview=FitH,            breaklinks=true,            colorlinks=true,            
citecolor=blue,            pagebackref=false]{hyperref}
\usepackage{subfiles}
\newtheorem{theorem}{Theorem} 
\newtheorem{proposition}{Proposition}
\newtheorem{corollary}{Corollary}
\newtheorem{definition}{Definition}
\newtheorem{lemma}{Lemma}
\newtheorem{example}{Example}
\newtheorem{assumption}{Assumption}
\newcommand{\blue}{\color{blue}}
\newcommand{\red}{\color{red}}
\newcommand{\redblue}{red!80!blue!}
\newcommand{\magblue}{magenta!60!blue!}
\newcommand{\JS}[1]{\authnote[blue]{JS}{#1}}
\newcommand{\RM}[1]{\authnote[teal]{RM}{#1}}
\newcommand{\LZ}[1]{\authnote[magenta]{LZ}{#1}}
\newcommand{\KS}[1]{\authnote[purple]{KS}{#1}}
%\textheight=23cm
%\usepackage{atbegshi}% http://ctan.org/pkg/atbegshi
%\AtBeginDocument{\AtBeginShipoutNext{\AtBeginShipoutDiscard}}

%\textwidth=15.7cm
\textwidth=16.5cm
\topmargin=-1cm
\oddsidemargin=0cm

\begin{document}

\title{\large\bf{\color{\redblue}
From Debt Sustainability Analysis to Liability Sustainability Analysis:\\
Making the European Welfare State Sustainable} \\{\color{gray}with ageing and increasing defence expenditures}\thanks{Preliminary draft, not to be quoted.}}
\author{
{\small\bf\color{blue}Ramon Marimon}\thanks{\scriptsize European University Institute, BSE, UPF, CREi, CEPR and NBER}
\and
{\small\bf\color{blue}Kamila Slawinska}\thanks{\scriptsize European Stability Mechanism}
\and
{\small\bf\color{blue}Jo\~oao Brogueira de Sousa}\thanks{\scriptsize Universidade NOVA de Lisboa}
\and
{\small\bf\color{blue}Luca Zavalloni}\thanks{\scriptsize European Stability Mechanism}}
\date{December 22, 2025}
%\href{ https://www.ramonmarimon.eu/abp-last/}{\color{blue}\small{Click Here for the Most Recent Version}}}
\maketitle

\begin{abstract}
\label{s:abstract}
Sovereign debt is one of the liabilities of a government, but the list of non-debt liabilities is large, including the sustainability of the welfare state and the provision of public goods. Debt Sustainability Analyses (DSA) compares government debt projections to the evolution of budget surpluses but does not account for the intertemporal trade-offs of fiscal policy, when the objective is the long-run sustainability of government liabilities. Liability Sustainability Analysis (LSA) is needed, but to understand the trade-offs and design efficient sustainable and counter-cyclical fiscal policies a stochastic dynamic general equilibrium model is also needed. This analysis is particularly relevant now that European countries, already ageing and indebted, face an historical military buildup. There is a consensus that choices of policies and reforms will need to be made to keep all liabilities sustainable (e.g. \cite{IMF202511}), but not about which and how, and their welfare consequences. In this draft, we develop, theoretically and computationally, a dynamic overlapping generation model with heterogeneous cohorts that can encompass different forms of the European Welfare State, as well as of accumulation of private and public capital. Calibrated to different European countries, this model-framework will allow us: first, to have a better picture of the European \textquoteleft liabilities experiences\textquoteright\, through the XXI$^{\rm{st}}$ crises; second, address the question of whether their current liabilities are sustainable (and their welfare costs) in face of the ageaing transition and the foreseen military buildup, and third, study alternative policies and reforms that can make liabilities -- including debt -- sustainable, a guarantee to maintain, and possibly improve, European citizens welfare.  
\end{abstract}

\thispagestyle{empty}

\clearpage

\setcounter{page}{1}

\section{Introduction}
\label{s:introduction}
\subfile{DSA-LSA introduction}

\section{A Sovereign Debt Model with Life-Cycle Households, Health Shocks, Pensions, Taxes, and Transfers}
\label{s:model}
\subfile{DSA-LSA model}

\section{Data}
\label{s:data}
\subfile{DSA-LSA data}

\section{Policy Experiments (Preliminary)}
\label{s:experiments}
\subfile{DSA-LSA experiments}

\vskip2em
\phantomsection
\addcontentsline{toc}{section}{References}
{\footnotesize
\setlength{\bibsep}{0ex}
\renewcommand{\baselinestretch}{1}
\bibliographystyle{ecta}
\bibliography{ref}
}

\begin{appendices}
\section{Calibration Strategy}
\label{s:calibration}
\subfile{DSA-LSA calibration}
\end{appendices}

\end{document}

\subsection{Environment}
\begin{itemize}[leftmargin=*]
    \item \textbf{Country:} small open economy (SOE) that issues non-state-contingent sovereign bonds to foreign lenders.
    \item \textbf{Households:} overlapping generations (OLG) with three life stages: \textbf{schooling}, \textbf{working}, \textbf{retirement}. Incomplete markets; idiosyncratic risks (labor productivity, employment, health). Survival is stochastic and age/health-dependent.
    \item \textbf{Government:} sets tax rates, pension rules, and transfers/health subsidies; issues sovereign bonds (one-period or long-term), invests in productive public capital. 
    {\RM With a special focus on: i) the European Welfare States: Education, Health and Social Security (Pensions, Unemployment,...), and ii) strategic expenditures/investments; in particular, defense.}
    \item \textbf{Firms/technology:} tradable/nontradable production structure with exogenous TFP shocks.
    \item \textbf{External lenders:} risk-neutral, discount at world rate $r^{*}$.
\end{itemize}


\subsection{Demographics \& Life Cycle}
Ages $j \in \{1,\dots,J\}$. Partition:
\begin{align*}
%\textbf{Schooling: } & j \in \mathcal{S}=\{1,\dots,J_S\}, \\
\textbf{Working: } & j \in \mathcal{W}=\{1,\dots,J_R\}, \\
\textbf{Retired: } & j \in \mathcal{R}=\{J_R+1,\dots,J\}.
\end{align*}
Survival probability $\pi(j,s)$ depends on age and health $s$. Newborn mass fills in to keep population stationary (or allow growth). Optionally, retirement is endogenous within a window $[J_R^{\min},J_R^{\max}]$.

\section{Preferences and Health}
Household $i$ derives period utility
\begin{equation}
u(c,\ell) = \frac{c^{1-\sigma}}{1-\sigma} - \nu\,\frac{\ell^{1+\varphi}}{1+\varphi},
\end{equation}
where $c$ is consumption, $\ell$ labor hours.

Health state $s \in \mathcal{S}$ follows $P_s(s'|s,j)$, affecting:
\begin{itemize}[leftmargin=*]
    \item labor efficiency,
    \item medical expenditure needs $m^{\text{need}}(j,s)$, \JS{should affect $P_s$ somehow}
    \item survival $\pi(j,s)$.
\end{itemize}

\section{Human Capital / Schooling}
Human capital has three components: (i) initial level determined by education expenditure, (ii) deterministic life-cycle growth, and (iii) stochastic shocks.

The initial human capital level depends on education expenditure:
\begin{equation}
    h_0 = H(e_s) = A_0 e_s^{\gamma},
\end{equation}
where $e_s$ is education expenditure, $A_0$ is initial productivity, and $\gamma \in (0,1)$ captures returns to education investment.

After entering the labor market, human capital evolves according to:
\begin{equation}
    \log h_{j+1} = \log h_j + g_j + \epsilon_j,
\end{equation}
where $g_j$ is a deterministic age-dependent growth rate capturing experience effects, and $\epsilon_j \sim iid \; N(0,\sigma_\epsilon^2)$ represents stochastic shocks to human capital.
{\RM we may want to consider (in the future?) $g_j$ also dependent on $e$.}

Human capital and health status map directly into wage units in working years through $w_t \cdot h_j \cdot f(s)$, where health status $s \in \{ s_{\text{very bad}}, s_{\text{good}}, s_{\text{bad}} \}$ affects labor productivity directly.

\textbf{References:} 
\begin{itemize}[leftmargin=*]
    \item Huggett, M., Ventura, G., \& Yaron, A. (2011). ``Sources of Lifetime Inequality.'' \emph{AER}.
    \item Caucutt, E. M., \& Lochner, L. (2020). ``Early and Late Human Capital Investments, Borrowing Constraints, and the Family.'' \emph{JPE}.
\end{itemize}

\section{Idiosyncratic Income \& Employment Risk}
Productivity $z_i \in \{0,z_1,...,z_N\}$ follows a Markov process $P_z(z'|z,j,s)$, where $z_i=0$ represents unemployment. Wage income in working years:
\begin{equation}
y^L = w_t\, h_j\, f(s) \, z.
\end{equation}
When $z=0$, the household receives unemployment benefits through the transfer system.

\section{Household Budget Constraints}\label{sec:household-bc}
Let $a_i$ denote individual \emph{liquid assets} (domestic private claims; prohibit households from borrowing externally). Let $\underline{a}$ be a borrowing limit (possibly $0$).

\subsection*{Schooling}
Children are part of the household, parents have to provide for consumption and for the fraction of education costs that is not covered by the government. Each cohort of working age household has an exogenous number of children during the first $\mathcal{S}$ of working life.

\subsection*{Working ($j \in \mathcal{W}$)}
\begin{equation}
(1+\tau^{c})(1 + c_y(j))c + a' + m^{\text{oop}} + e_s = (1-\tau^{l}-\tau^{p})\, y^L\,\ell + T^{W}(\cdot) + (1+r^d_t)a,
\end{equation}
with consumption tax $\tau^c$, labor income tax $\tau^l$, and \textbf{social security contributions} $\tau^p$ that finances social security; $T^{W}$ are means-tested transfers; $m^{\text{oop}} = \big(1-\kappa^{\text{health}}_t(j,s)\big)\, m^{\text{need}}(j,s)$ with public health coverage rate $\kappa^{\text{health}}_t$. $c_y(j)$ is the fraction of household consumption that is spent by children during schooling years, and $e_s = \big(1-\kappa^{\text{school}}_t(j)\big)\, e(j)$ are education expenditures.
{\RM Note that $\kappa^{\text{health}}_t(j) \& \kappa^{\text{school}}_t(j)$ parameterize two important pillars of the Esuropean Welfare States. [Better change the notation $c_y(j)$, keeping $c$ for consumption].}

 
\subsection*{{\RM{Unemployed ($j \in \mathcal{U}$)}}}
{\RM {we may want to consider them, even if the transition to in-and-out of $\mathcal{U})$ are -- for the time being -- exogenous, but this should be part of the individual state and keep account of the unemployment benefits expenditures.}}

\subsection*{Retired ($j \in \mathcal{R}$)}
\begin{equation}
c + a' + \tau^{c} c + m^{\text{oop}} = \text{PENS}_t\big(\bar{y}_t,j\big) + T^{R}(\cdot) + (1+r^d_t)a,
\end{equation}
where $\text{PENS}_t(\bar{y}_t,a)$ is a benefit function (e.g., replacement rate $\rho$ applied to lifetime average earnings $\bar{y}_t$, with floors/caps; indexation rules can be to prices or wages).
{\RA {Be more explicit about the distinction PAYG vs Fully Funded (or partially Funded. However, the latter will require to keep track of the individual contributions (not $a$).}}
\subsection*{Terminal Condition}
Assets of the deceased are taxed at rate $\tau^{\text{beq}}$ and redistributed lump-sum.
{\RM {To whom?}}

\section{Production}

% A representative firm with Cobb-Douglas technology:
% \begin{equation}
%     Y_t = Z_t \left(K^g_t\right)^{\eta_g} K_t^{\alpha} L_t^{1-\alpha},
% \end{equation}
% where $Z_t$ is aggregate productivity, $K^g_t$ is public capital, $K_t$ is private capital, $L_t$ is aggregate labor, $\eta_g$ measures the output elasticity of public capital, and $\alpha$ is the private capital share.
    
% The firm hires labor at wage $w_t$; private capital $K_t$ can be fixed or pinned down by $r^{*}$. 

\subsection{Baseline}

The Cobb--Douglas production function is
\begin{equation}
Y = Z_t \left(K^g_t\right)^{\eta_g} K_t^{\alpha} L_t^{1-\alpha},
\end{equation}
where
\begin{itemize}
    \item $K^g$ : public capital,
    \item $K$ : private capital,
    \item $L$ : labor,
    \item $\alpha$ : elasticity of output with respect to private capital,
    \item $\eta_g$ : elasticity of output with respect to public capital.
\end{itemize}

\subsection{Generalized formulation with substitution}

To allow for imperfect substitution between $K$ and $K^g$ while keeping the
capital--labor split fixed, define an \emph{effective capital stock} $\tilde K(\sigma)$ as a CES aggregator:
\begin{equation}
\tilde K(\sigma) =
\left[ (1-\theta) K^{\rho} + \theta (K^g)^{\rho} \right]^{1/\rho},
\qquad
\rho = \frac{\sigma - 1}{\sigma},
\qquad
\sigma \in (0, \infty],
\end{equation}
where $\sigma$ is the elasticity of substitution between $K$ and $K_g$,
and $\theta \in (0,1)$ is the share parameter
\[
\theta = \frac{\eta}{\alpha + \eta}.
\]

The production function becomes
\begin{equation}
Y = Z\, \tilde K(\sigma)^{\alpha+\eta}\, L^{\,1-\alpha}.
\end{equation}

\subsection{Limiting cases}

\paragraph{Cobb--Douglas case ($\sigma \to 1$):}
When $\sigma \to 1$ (i.e.\ $\rho \to 0$),
\[
\tilde K \to K^{1-\theta} (K^g)^{\theta},
\]
so that
\[
Y \to Z\, K^{(\alpha+\eta)(1-\theta)} (K^g)^{(\alpha+\eta)\theta} L^{1-\alpha}
    = Z\, (K^g)^{\eta} K^{\alpha}  L^{1-\alpha}.
\]
Hence the original Cobb--Douglas form is recovered.

\paragraph{Perfect substitutes ($\sigma \to \infty$):}
As $\sigma \to \infty$ ($\rho \to 1$),
\[
\tilde K \to (1-\theta) K + \theta K^g,
\]
so that
\[
Y = Z\, \big[(1-\theta) K + \theta K^g\big]^{\alpha} L^{1-\alpha}.
\]

\paragraph{Perfect complements ($\sigma \to 0^+$):}
As $\sigma \to 0^+$, the aggregator becomes Leontief:
\[
\tilde K \to \min\{K, K^g\},
\]
so that
\[
Y = Z\, [\min\{K, K^g\}]^{\alpha} L^{1-\alpha}.
\]

\subsection{Remarks}

\begin{itemize}
    \item The elasticity of output with respect to labor remains fixed at $1-\alpha$.
    \item The elasticity with respect to the capital composite $\tilde K$ is $\alpha+\eta$.
    \item Substitutability between public and private capital is governed by $\sigma$:
    higher $\sigma$ implies greater substitutability, and $\sigma \to \infty$ yields perfect substitutes.
    \item The Cobb--Douglas specification is a special case of this CES formulation when $\sigma=1$.
\end{itemize}

\vspace{15pt}

\textbf{References to check:} 
\begin{itemize}[leftmargin=*]
    \item  Baxter and Barro (1993) "Fiscal Policy in General Equilibrium"
    \item Barro, R. J. (1990). "Government Spending in a Simple Model of Endogeneous Growth." \emph{Journal of Political Economy}, 98(5), S103-S125.
    \item Leeper, E. M., Walker, T. B., \& Yang, S. S. (2010). "Government Investment and Fiscal Stimulus." \emph{Journal of Monetary Economics}, 57(8), 1000-1012.
    \item Ramey (2020) "The Macroeconomic Consequences of
Infrastructure Investment"
    \item Chatterjee, S., \& Turnovsky, S. J. (2012). "Infrastructure and Inequality." \emph{European Economic Review}, 56(8), 1730-1745.
    \item Others ? 
  \end{itemize}

\section{Individual Problem}\label{sec:state}
\textbf{Individual state:} $(j, h, s, z, \bar{y}, a)$ and pension status.


\noindent \textbf{Aggregate state:} $\mathbf{S}_t = \big(r^*,B_t, Z_t, \mu_t\big)$, where $\mu_t$ is the \emph{distribution of households} over individual states.


\paragraph{Household Bellman (working age).}
\begin{align*}
V_j(h, s, z, \bar{y}, a;\,\mathbf{S}) = \max_{c,\ell,a' \ge \underline{a}} \;& u(c,\ell) + \beta\,\pi_j(s)\, \mathbb{E}\!\left[V_{j+1}(h', s', z', \bar{y}', a';\,\mathbf{S}')\right] \\
\text{s.t. } & (1+\tau^{c})(1 + c_y(j))\,c + a' + m^{\text{oop}} + e_s = \\
& \qquad (1-\tau^{l}-\tau^{p})\, w_t\, h_j\, f(s)\, z\, \ell + T^{W}(\cdot) + (1+r^d_t)\, a, \\
& \log h' = \log h + g_j + \sigma_{\epsilon}\epsilon', \\
& s' \sim P_s(s'|s,j), \quad z' \sim P_z(z'|z,j,s), \quad \epsilon' \sim N(0,1)\\
& \text{and government policies }. 
\end{align*}

\paragraph{Household Bellman (retired).}
\begin{align*}
V_j^{R}(s, \bar{y}, a;\,\mathbf{S}) = \max_{c,\,a' \ge \underline{a}} \;& u(c,0) + \beta\,\pi_j(s)\, \mathbb{E}\!\left[V_{j+1}^{R}(s', \bar{y}', a';\,\mathbf{S}')\right] \\
\text{s.t. } & (1+\tau^{c})\,c + a' + m^{\text{oop}} = \text{PENS}_t(\bar{y},j) + T^{R}(\cdot) + (1+r^d_t)\,a, \\
& m^{\text{oop}} = \big(1-\kappa^{\text{health}}_t(j,s)\big)\, m^{\text{need}}(j,s), \\
& s' \sim P_s(s'|s,j), \\
& \text{and government policies }, \\
& \bar{y}' = \bar{y} \quad \text{(no new earnings after retirement)}.
\end{align*}

{\RM {Recall the accounting of the unemployed}}

\section{Pensions, Taxes, and Transfers (Parameterizations)}
\begin{itemize}[leftmargin=*]
    \item \textbf{Pensions:} $\text{PENS}_t = \max\{\rho \cdot \bar{y}, \; b_{\min}\}$, $\rho$ is a replacement rate and $b_{i\min}$ a minimum pension, set by the government.

    \item \textbf{Taxes:} proportional rates $(\tau^c,\tau^l,\tau^p)$ or a piecewise-linear/progressive schedule $\tau^l(y) = 1 - \kappa y^{-\eta}$. Payroll taxes fund pensions (pay-as-you-go), optionally with a trust-fund asset $S^{\text{pens}}_t$.
    \item \textbf{Transfers:} age- and means-tested floors for students/workers/retirees; unemployment subsidies; health subsidies $\kappa^{\text{health}}_t(a,h)$.
\end{itemize}

\section{Aggregate Resource Constraint}

Define aggregate uses of resources:
\begin{align*}
C_t &\equiv \int (1 + c_y(j))\, c(j,\cdot)\, d\mu_t, \\
M_t &\equiv \int m^{\text{need}}(j,s)\, d\mu_t, \\
E_t &\equiv \int_{\mathcal{W}} e(j)\, d\mu_t, \\
I^k_t &\text{ s.t. } K_{t+1} = (1-\delta_k)K_t + I^k_t, \\
I^g_t &\text{ s.t. } K^g_{t+1} = (1-\delta_g)K^g_t + I^g_t.
\end{align*}

Aggregate output is given by:
\begin{equation}
Y_t = Z_t \left(K^g_t\right)^{\eta_g} K_t^{\alpha} L_t^{1-\alpha}.
\end{equation}

\subsection*{One-Period External Bonds}
The economy-wide resource constraint (goods-market clearing condition) is:
\begin{equation}
C_t + M_t + E_t + G_t + I^k_t + I^g_t = Y_t + q_t B_{t+1} - (1+r^{*})B_t.
\end{equation}

\section{Government}
\paragraph{Instruments.} Debt $B_{t+1}$ (external sovereign bonds, one-period or long-term); tax schedule $(\tau^c_t,\tau^l_t,\tau^p_t)$; public investment $I^g_t$; spending $G_t$; transfers $T^{W},T^{R}$; health coverage $\kappa^{\text{health}}_t(j,s)$; education subsidies $\kappa^{\text{school}}_t(j)$; pension rules with replacement rate $\rho$, retirement age $J_R$, and indexation.

\paragraph{Flow budget.} The government budget constraint incorporates all spending components:

For \emph{one-period bonds}:
\begin{align}
& G_t + I^g_t + (1+r^{*})B_t + \int_{\mathcal{R}} \text{PENS}_t(\bar{y}_t,j)\,d\mu_t + \nonumber \\
& \quad + \int \kappa^{\text{health}}_t(j,s)\,m^{\text{need}}(j,s)\,d\mu_t + \int_{\mathcal{W}} \kappa^{\text{school}}_t(j)\,e(j)\,d\mu_t + \int T^W(\cdot)\,d\mu_t + \int T^R(\cdot)\,d\mu_t \nonumber \\
& = \tau^c_t \int (1+c_y(j))c\,d\mu_t + (\tau^l_t + \tau^p_t) \int_{\mathcal{W}} w_t h_j s z \ell\,d\mu_t +  B_{t+1} + \tau^{\text{beq}} \int (1-\pi(j,s))a\,d\mu_t,
\end{align}

\noindent where public capital evolves according to:
\begin{equation}
K^g_{t+1} = (1-\delta_g)K^g_t + I^g_t.
\end{equation}

For \emph{long-term bonds}, replace $(1+r^*)B_t$ with $\delta B_t$ in the above expression, where $\delta$ represents the maturing fraction plus coupons.

\section{GBC: Aggregation}
Define tax bases and spending aggregates using the cross-sectional distribution $\mu_t$ from Section~\ref{sec:state} and Section 13:
\begin{align*}
C_t^{\text{base}} &\equiv \int (1+c_y(j))\, c(x,j)\, d\mu_t, \\
\mathcal{B}_t^{\text{lab}} &\equiv \int_{j\in \mathcal{W}} w_t\, h_j\, f(s)\, z\, \ell(x,j)\, d\mu_t, \\
\text{PENS}_t^{\text{out}} &\equiv \int_{j\in \mathcal{R}} \text{PENS}_t(\bar{y},j)\, d\mu_t, \\
\text{HSub}_t &\equiv \int \kappa^{\text{health}}_t(j,s)\, m^{\text{need}}(j,s)\, d\mu_t, \\
\text{ESub}_t &\equiv \int_{j\in \mathcal{W}} \kappa^{\text{school}}_t(j)\, e(j)\, d\mu_t, \\
\text{Tr}_t^W &\equiv \int T^W(\cdot)\, d\mu_t, \qquad
\text{Tr}_t^R \equiv \int T^R(\cdot)\, d\mu_t, \\
\text{BeqTax}_t &\equiv \tau^{\text{beq}} \int (1-\pi(j,s))\, a(x,j)\, d\mu_t.
\end{align*}
Government revenues:
\begin{align*}
\text{Rev}_t^c &= \tau_t^c \, C_t^{\text{base}}, \\
\text{Rev}_t^l &= \tau_t^l \, \mathcal{B}_t^{\text{lab}}, \qquad
\text{Rev}_t^p = \tau_t^p \, \mathcal{B}_t^{\text{lab}}.
\end{align*}
Spending includes $G_t$ (non-transfer purchases) and $I_t^g$ (public investment), with
$K^g_{t+1} = (1-\delta_g)K^g_t + I_t^g$.

Flow budget with one-period external bonds:
\begin{equation}
G_t + I_t^g + (1+r^{*})B_t + \text{PENS}_t^{\text{out}} + \text{HSub}_t + \text{ESub}_t + \text{Tr}_t^W + \text{Tr}_t^R
= \text{Rev}_t^c + \text{Rev}_t^l + \text{Rev}_t^p + \text{BeqTax}_t +  B_{t+1}.
\label{eq:gov-budget-1p}
\end{equation}
With long-term bonds (decay $\delta$), replace the debt service/issuance terms by
\begin{equation}
\underbrace{\delta B_t}_{\text{coupons+principal paid}} \quad\text{and}\quad \underbrace{ B_{t+1} - (1-\delta)  B_t}_{\text{net issuance at market value}},
\end{equation}
so \eqref{eq:gov-budget-1p} becomes
\begin{align}
G_t + I_t^g + \delta B_t + \text{PENS}_t^{\text{out}} + \text{HSub}_t + \text{ESub}_t + \text{Tr}_t^W + \text{Tr}_t^R
= \text{Rev}_t^c + \text{Rev}_t^l + \text{Rev}_t^p + \text{BeqTax}_t +  B_{t+1} - (1-\delta)  B_t.
\end{align}

Optional pension trust fund $S_t^{\text{pens}}$ (if used) accumulates as
\begin{equation}
S_{t+1}^{\text{pens}} = (1+r^{*}) S_t^{\text{pens}} + \text{Rev}_t^p - \text{PENS}_t^{\text{out}},
\end{equation}
and pension flows can be excluded from \eqref{eq:gov-budget-1p} accordingly.

{\RM{ I think that in any case we need to be more explicit about what $\text{Rev}_t^p$ covers with possibly differentiating the part that goes to $\text{PENS}_t^{\text{out}}$ and, as said, the individual contributions to the pension trust should be part of the individual state, no?}}

\section{Aggregate Resource Constraint and Market Clearing}
Aggregate uses:
\begin{align*}
C_t &\equiv \int (1+c_y(j))\, c(x,j)\, d\mu_t, \\
M_t &\equiv \int m^{\text{need}}(j,s)\, d\mu_t, \\
E_t &\equiv \int_{j\in \mathcal{W}} e(j)\, d\mu_t, \\
I_t^k &\text{ with } K_{t+1} = (1-\delta_k)K_t + I_t^k, \qquad
I_t^g \text{ with } K^g_{t+1} = (1-\delta_g)K^g_t + I_t^g.
\end{align*}
Aggregate effective labor and output:
\begin{align*}
L_t &\equiv \int_{j\in \mathcal{W}} h_j\, f(s)\, z\, \ell(x,j)\, d\mu_t, \\
Y_t &= Z_t \left(K_t^g\right)^{\eta_g} K_t^{\alpha} L_t^{1-\alpha}
\quad \text{(or the CES version if used).}
\end{align*}

Goods-market clearing with one-period bonds:
\begin{equation}
C_t + M_t + E_t + G_t + I_t^k + I_t^g = Y_t +  B_{t+1} - (1+r^{*}) B_t.
\end{equation}
With long-term bonds (decay $\delta$):
\begin{equation}
C_t + M_t + E_t + G_t + I_t^k + I_t^g
= Y_t + \big[ B_{t+1} - (1-\delta)  B_t\big] - \delta B_t.
\end{equation}

Market clearing:
\begin{align*}
&\text{Labor: } L_t = \int_{j\in \mathcal{W}} h_j f(s) z\, \ell\, d\mu_t, \\
&\text{Capital: } K_{t+1} = (1-\delta_k)K_t + I_t^k, \quad K^g_{t+1} = (1-\delta_g)K_t^g + I_t^g, \\
&\text{External bond: } \text{foreign lenders hold } B_t \text{ supplied by the government.}
\end{align*}

\section{Overlapping Generations}
Let $j\in\{1,\dots,J\}$ denote age. For each age $j$, let $\mathcal{X}_j$ be the individual state space \emph{excluding age} (e.g., $x=(h,s,z,\bar y,a,\text{pension})$), and let $\mu_{t,j}$ be a Borel measure over $\mathcal{X}_j$ at date $t$. Define the disjoint-union state space
\[
\mathcal{X} \equiv  \bigcup_{j=1}^J \big(\mathcal{X}_j \times \{j\}\big),
\]
so the \emph{full} individual state is $(x,j)$ with $x\in\mathcal{X}_j$. For any measurable $A \subseteq \mathcal{X}$, define the cross-sectional distribution as
\[
\mu_t(A) \equiv \sum_{j=1}^J \mu_{t,j}\big(\{x\in\mathcal{X}_j : (x,j)\in A\}\big),
\]
so that $N_t \equiv \mu_t(\mathcal{X}) = \sum_{j=1}^J \mu_{t,j}(\mathcal{X}_j)$.

\paragraph{Age progression with survival.}
Let $x\in\mathcal{X}_j$ denote the age-excluding individual state and let $P_j(\cdot\mid x)$ be the one-period transition kernel for $x'$ at next age $j+1$ (induced by model primitives and optimal choices). Survival depends on $(j,s)$ via $\pi(j,s)$. For any measurable $A\subseteq\mathcal{X}_{j+1}$ and ages $j=1,\dots,J-1$,
\begin{equation}
\mu_{t+1,j+1}(A) \;=\; \int_{\mathcal{X}_j} \pi(j,s)\, P_j(A\mid x)\; \mu_{t,j}(dx).
\label{eq:mu-age-progression}
\end{equation}
Agents at terminal age $J$ exit the economy after $t$; there is no $j=J+1$ cohort.

\paragraph{Births and population growth.}
Let $\Lambda \equiv 1+g_N$ be the exogenous gross population growth factor (set $g_N=0$ for stationary population). Let $\psi$ be the initial-state distribution of newborns over $\mathcal{X}_1$ (e.g., initial health $s$ with probabilities $\psi_s$, initial assets $a=0$, initial human capital $h_0=H(e_s)$, etc.). Define births mass $B_{t+1}$ to satisfy the total population law of motion:
\begin{align}
\mu_{t+1,1}(A) &= B_{t+1}\, \psi(A), \qquad \forall A\subseteq\mathcal{X}_1, \label{eq:newborn-dist}\\
N_{t+1} &= \Lambda\, N_t \;=\; \underbrace{\sum_{j=1}^{J-1} \int_{\mathcal{X}_j} \pi(j,s)\,\mu_{t,j}(dx)}_{\text{survivors to }t+1} \;+\; B_{t+1}. \label{eq:population-total}
\end{align}
Equivalently,
\begin{equation}
B_{t+1} \;=\; \Lambda\, N_t \;-\; \sum_{j=1}^{J-1} \int_{\mathcal{X}_j} \pi(j,s)\,\mu_{t,j}(dx).
\end{equation}
With $g_N=0$, births replace deaths to keep $N_{t+1}=N_t$.

\paragraph{Cohort-year margins and health decomposition.}
Let $N_{t,j}\equiv \mu_{t,j}(\mathcal{X}_j)$ be the mass of age-$j$ agents in year $t$.
If we track only age and health, define $N_{t,j,s}\equiv \mu_{t,j}(\{x\in\mathcal{X}_j:\text{health}=s\})$, then for $j=1,\dots,J-1$:
\begin{align}
N_{t+1,j+1,s'} &= \sum_{s} \pi(j,s)\, P_s(s'|s,j)\, N_{t,j,s}, \\
N_{t,j} &= \sum_{s} N_{t,j,s}, \qquad N_t = \sum_{j=1}^J N_{t,j}.
\end{align}

\paragraph{Aggregation by cohort-year.}
For any measurable individual quantity $x_t(x,j)$ (e.g., consumption, labor, medical needs),
define its cohort-year aggregate and economy-wide aggregate by
\begin{align}
X_{t,j} &\equiv \int_{\mathcal{X}_j} x_t(x,j)\; \mu_{t,j}(dx), \\
X_t &\equiv \sum_{j=1}^J X_{t,j} \;=\; \sum_{j=1}^J \int_{\mathcal{X}_j} x_t(x,j)\; \mu_{t,j}(dx) \;=\; \int_{\mathcal{X}} x_t(x,j)\; d\mu_t.
\end{align}
The integrals appearing in the government budget constraint and resource constraint are special cases of $X_t$ with appropriate choices of $x_t(\cdot)$ and domain restrictions (e.g., $j\in\mathcal{W}$ for working-age terms).

\end{document}
